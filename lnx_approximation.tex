\documentclass[11pt,a4paper]{article}
\usepackage[utf8]{inputenc}
\usepackage[T1]{fontenc}
\usepackage[french]{babel}
\usepackage{amsmath, amssymb}
\usepackage{geometry}
\geometry{margin=2cm}
\begin{document}
\section*{Approximation de ln(x) par des suites}
\subsection*{Question 1: Prouver les inégalités $T(x) \le 2T(\sqrt{x})$ et $2S(\sqrt{x}) \le S(x)$ pour tout $x \in [1, +\infty[$.}
\paragraph*{Étape 1:} Substitution des définitions de $T(x)$ et $T(\sqrt{x})$.
\[ $T(x) \le 2T(\sqrt{x}) \iff \frac{x^2 - 1}{x^2 + 1} \le 2 \frac{x - 1}{x + 1}$ \]
\paragraph*{Étape 2:} Factorisation de $x^2-1$ et simplification pour $x > 1$.
\[ $\frac{(x-1)(x+1)}{x^2 + 1} \le 2 \frac{x - 1}{x + 1} \iff \frac{x+1}{x^2 + 1} \le \frac{2}{x + 1}$ \]
\paragraph*{Étape 3:} Réarrangement algébrique.
\[ $(x+1)^2 \le 2(x^2 + 1)$ \]
\paragraph*{Étape 4:} Développement et simplification.
\[ $x^2 + 2x + 1 \le 2x^2 + 2 \iff 0 \le x^2 - 2x + 1$ \]
\paragraph*{Étape 5:} Identité remarquable.
\[ $0 \le (x-1)^2$, ce qui est vrai pour tout $x \in [1, +\infty[$. \]
\paragraph*{Étape 6:} Substitution des définitions de $S(x)$ et $S(\sqrt{x})$.
\[ $2S(\sqrt{x}) \le S(x) \iff 2 \frac{x - 1}{2\sqrt{x}} \le \frac{x^2 - 1}{2x}$ \]
\paragraph*{Étape 7:} Simplification et factorisation de $x^2-1$.
\[ $\frac{x - 1}{\sqrt{x}} \le \frac{(x-1)(x+1)}{2x}$ \]
\paragraph*{Étape 8:} Réarrangement algébrique pour $x > 1$.
\[ $2x \le \sqrt{x}(x+1) \iff 2\sqrt{x} \le x+1$ \]
\paragraph*{Étape 9:} Réarrangement.
\[ $0 \le x - 2\sqrt{x} + 1$ \]
\paragraph*{Étape 10:} Identité remarquable.
\[ $0 \le (\sqrt{x} - 1)^2$, ce qui est vrai pour tout $x \in [1, +\infty[$. \]
\paragraph*{Résultat:} $\text{N/A}$
\end{document}