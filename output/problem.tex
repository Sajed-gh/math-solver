\documentclass{article}
\usepackage{amsmath, amssymb, geometry, xcolor}
\geometry{margin=1in}
\begin{document}
\section*{Approximation de ln(x) par des suites}

\subsection*{Question I.1}
\textbf{Givens:}
\begin{itemize}
  \item $x \textbackslash{}in [1, +\textbackslash{}infty[$
  \item $T(x) = \textbackslash{}frac\textbackslash{}{x\textbackslash{}textasciicircum{}2 - 1\textbackslash{}}\textbackslash{}{x\textbackslash{}textasciicircum{}2 + 1\textbackslash{}}$
  \item $S(x) = \textbackslash{}frac\textbackslash{}{x\textbackslash{}textasciicircum{}2 - 1\textbackslash{}}\textbackslash{}{2x\textbackslash{}}$
\end{itemize}
\textbf{Requested:} $T(x) \textbackslash{}le 2T(\textbackslash{}sqrt\textbackslash{}{x\textbackslash{}})$ et $2S(\textbackslash{}sqrt\textbackslash{}{x\textbackslash{}}) \textbackslash{}le S(x)$

\textbf{Strategy Outline:}
\begin{enumerate}
  \item Substitute $\textbackslash{}sqrt\textbackslash{}{x\textbackslash{}}$ into $T(x)$ and $S(x)$ definitions.
  \item Simplify $2T(\textbackslash{}sqrt\textbackslash{}{x\textbackslash{}})$ and $2S(\textbackslash{}sqrt\textbackslash{}{x\textbackslash{}})$.
  \item Prove the inequalities by algebraic manipulation, showing they reduce to $(x-1)\textbackslash{}textasciicircum{}2 \textbackslash{}ge 0$ or $(\textbackslash{}sqrt\textbackslash{}{x\textbackslash{}}-1)\textbackslash{}textasciicircum{}2 \textbackslash{}ge 0$.
\end{enumerate}
\textbf{Detailed Steps:}
\begin{enumerate}
  \item \textbf{Justification:} Substitution de $\textbackslash{}sqrt\textbackslash{}{x\textbackslash{}}$ dans $T(x)$ et simplification \\
        \textbf{Expression:} $2T(\sqrt{x}) = 2 \frac{(\sqrt{x})^2 - 1}{(\sqrt{x})^2 + 1} = \frac{2(x-1)}{x+1}$
  \item \textbf{Justification:} Énoncé de l'inégalité à prouver \\
        \textbf{Expression:} $T(x) \le 2T(\sqrt{x}) \iff \frac{x^2 - 1}{x^2 + 1} \le \frac{2(x-1)}{x+1}$
  \item \textbf{Justification:} Factorisation de $x\textbackslash{}textasciicircum{}2-1$ \\
        \textbf{Expression:} $ \frac{(x-1)(x+1)}{x^2 + 1} \le \frac{2(x-1)}{x+1}$
  \item \textbf{Justification:} Division par $(x-1)$ (valide pour $x>1$, trivial pour $x=1$) \\
        \textbf{Expression:} $ \frac{x+1}{x^2 + 1} \le \frac{2}{x+1}$ pour $x > 1$
  \item \textbf{Justification:} Multiplication croisée \\
        \textbf{Expression:} $(x+1)^2 \le 2(x^2 + 1)$
  \item \textbf{Justification:} Développement \\
        \textbf{Expression:} $x^2 + 2x + 1 \le 2x^2 + 2$
  \item \textbf{Justification:} Réarrangement algébrique \\
        \textbf{Expression:} $0 \le x^2 - 2x + 1$
  \item \textbf{Justification:} Factorisation, toujours vrai \\
        \textbf{Expression:} $0 \le (x-1)^2$
  \item \textbf{Justification:} Substitution de $\textbackslash{}sqrt\textbackslash{}{x\textbackslash{}}$ dans $S(x)$ et simplification \\
        \textbf{Expression:} $2S(\sqrt{x}) = 2 \frac{(\sqrt{x})^2 - 1}{2\sqrt{x}} = \frac{x-1}{\sqrt{x}}$
  \item \textbf{Justification:} Énoncé de l'inégalité à prouver \\
        \textbf{Expression:} $2S(\sqrt{x}) \le S(x) \iff \frac{x-1}{\sqrt{x}} \le \frac{x^2 - 1}{2x}$
  \item \textbf{Justification:} Factorisation de $x\textbackslash{}textasciicircum{}2-1$ \\
        \textbf{Expression:} $ \frac{x-1}{\sqrt{x}} \le \frac{(x-1)(x+1)}{2x}$
  \item \textbf{Justification:} Division par $(x-1)$ (valide pour $x>1$, trivial pour $x=1$) \\
        \textbf{Expression:} $ \frac{1}{\sqrt{x}} \le \frac{x+1}{2x}$ pour $x > 1$
  \item \textbf{Justification:} Multiplication croisée \\
        \textbf{Expression:} $2x \le \sqrt{x}(x+1)$
  \item \textbf{Justification:} Division par $\textbackslash{}sqrt\textbackslash{}{x\textbackslash{}}$ \\
        \textbf{Expression:} $2\sqrt{x} \le x+1$
  \item \textbf{Justification:} Réarrangement algébrique \\
        \textbf{Expression:} $0 \le x - 2\sqrt{x} + 1$
  \item \textbf{Justification:} Factorisation, toujours vrai \\
        \textbf{Expression:} $0 \le (\sqrt{x}-1)^2$
\end{enumerate}
\textbf{Final Answer:} \boxed{$\text{Les inégalités sont prouvées.}$}

\subsection*{Question I.2}
\textbf{Givens:}
\begin{itemize}
  \item $f(x) = \textbackslash{}ln(x) - T(x)$
  \item $g(x) = S(x) - \textbackslash{}ln(x)$
  \item $T(x) = \textbackslash{}frac\textbackslash{}{x\textbackslash{}textasciicircum{}2 - 1\textbackslash{}}\textbackslash{}{x\textbackslash{}textasciicircum{}2 + 1\textbackslash{}}$
  \item $S(x) = \textbackslash{}frac\textbackslash{}{x\textbackslash{}textasciicircum{}2 - 1\textbackslash{}}\textbackslash{}{2x\textbackslash{}}$
  \item $x \textbackslash{}in [1, +\textbackslash{}infty[$
\end{itemize}
\textbf{Requested:} a) $f'(x) = \textbackslash{}frac\textbackslash{}{(x\textbackslash{}textasciicircum{}2 - 1)\textbackslash{}textasciicircum{}2\textbackslash{}}\textbackslash{}{x(x\textbackslash{}textasciicircum{}2 + 1)\textbackslash{}textasciicircum{}2\textbackslash{}}$ et $g'(x) = \textbackslash{}frac\textbackslash{}{(x-1)\textbackslash{}textasciicircum{}2\textbackslash{}}\textbackslash{}{2x\textbackslash{}textasciicircum{}2\textbackslash{}}$ ; b) $0 \textbackslash{}le f'(x) \textbackslash{}le (x-1)\textbackslash{}textasciicircum{}2$ et $0 \textbackslash{}le g'(x) \textbackslash{}le \textbackslash{}frac\textbackslash{}{1\textbackslash{}}\textbackslash{}{2\textbackslash{}}(x-1)\textbackslash{}textasciicircum{}2$

\textbf{Strategy Outline:}
\begin{enumerate}
  \item Calculate the derivatives $f'(x)$ and $g'(x)$ using quotient rule and chain rule.
  \item Simplify the derivatives to match the given expressions.
  \item For part b), analyze the sign of the derivatives and prove the upper bounds by algebraic manipulation.
\end{enumerate}
\textbf{Detailed Steps:}
\begin{enumerate}
  \item \textbf{Justification:} Définition de $f'(x)$ \\
        \textbf{Expression:} $f'(x) = \frac{d}{dx}(\ln(x) - \frac{x^2 - 1}{x^2 + 1})$
  \item \textbf{Justification:} Application des règles de dérivation \\
        \textbf{Expression:} $f'(x) = \frac{1}{x} - \frac{2x(x^2+1) - (x^2-1)(2x)}{(x^2+1)^2}$
  \item \textbf{Justification:} Simplification du numérateur \\
        \textbf{Expression:} $f'(x) = \frac{1}{x} - \frac{2x^3 + 2x - 2x^3 + 2x}{(x^2+1)^2} = \frac{1}{x} - \frac{4x}{(x^2+1)^2}$
  \item \textbf{Justification:} Mise au même dénominateur et développement \\
        \textbf{Expression:} $f'(x) = \frac{(x^2+1)^2 - 4x^2}{x(x^2+1)^2} = \frac{x^4 + 2x^2 + 1 - 4x^2}{x(x^2+1)^2}$
  \item \textbf{Justification:} Factorisation du numérateur \\
        \textbf{Expression:} $f'(x) = \frac{x^4 - 2x^2 + 1}{x(x^2+1)^2} = \frac{(x^2 - 1)^2}{x(x^2 + 1)^2}$
  \item \textbf{Justification:} Définition de $g'(x)$ \\
        \textbf{Expression:} $g'(x) = \frac{d}{dx}(\frac{x^2 - 1}{2x} - \ln(x))$
  \item \textbf{Justification:} Application des règles de dérivation \\
        \textbf{Expression:} $g'(x) = \frac{1}{2} \frac{2x(x) - (x^2-1)(1)}{x^2} - \frac{1}{x}$
  \item \textbf{Justification:} Simplification du numérateur \\
        \textbf{Expression:} $g'(x) = \frac{1}{2} \frac{2x^2 - x^2 + 1}{x^2} - \frac{1}{x} = \frac{x^2 + 1}{2x^2} - \frac{1}{x}$
  \item \textbf{Justification:} Mise au même dénominateur et factorisation du numérateur \\
        \textbf{Expression:} $g'(x) = \frac{x^2 + 1 - 2x}{2x^2} = \frac{(x-1)^2}{2x^2}$
  \item \textbf{Justification:} Le numérateur est un carré, le dénominateur est positif pour $x \textbackslash{}ge 1$ \\
        \textbf{Expression:} $f'(x) = \frac{(x^2 - 1)^2}{x(x^2 + 1)^2} \ge 0$
  \item \textbf{Justification:} Énoncé de l'inégalité à prouver \\
        \textbf{Expression:} $f'(x) \le (x-1)^2 \iff \frac{(x^2 - 1)^2}{x(x^2 + 1)^2} \le (x-1)^2$
  \item \textbf{Justification:} Factorisation de $x\textbackslash{}textasciicircum{}2-1$ \\
        \textbf{Expression:} $ \frac{(x-1)^2(x+1)^2}{x(x^2 + 1)^2} \le (x-1)^2$
  \item \textbf{Justification:} Division par $(x-1)\textbackslash{}textasciicircum{}2$ (valide pour $x>1$, trivial pour $x=1$) \\
        \textbf{Expression:} $ \frac{(x+1)^2}{x(x^2 + 1)^2} \le 1$ pour $x > 1$
  \item \textbf{Justification:} Multiplication par le dénominateur \\
        \textbf{Expression:} $(x+1)^2 \le x(x^2 + 1)^2$
  \item \textbf{Justification:} Développement \\
        \textbf{Expression:} $x^2 + 2x + 1 \le x(x^4 + 2x^2 + 1) = x^5 + 2x^3 + x$
  \item \textbf{Justification:} Réarrangement algébrique. La fonction $h(x) = x(x\textbackslash{}textasciicircum{}2+1)\textbackslash{}textasciicircum{}2 - (x+1)\textbackslash{}textasciicircum{}2$ a $h(1)=0$ et $h'(x) > 0$ pour $x \textbackslash{}ge 1$, donc $h(x) \textbackslash{}ge 0$. \\
        \textbf{Expression:} $0 \le x^5 + 2x^3 - x^2 - x - 1$
  \item \textbf{Justification:} Le numérateur est un carré, le dénominateur est positif pour $x \textbackslash{}ge 1$ \\
        \textbf{Expression:} $g'(x) = \frac{(x-1)^2}{2x^2} \ge 0$
  \item \textbf{Justification:} Énoncé de l'inégalité à prouver \\
        \textbf{Expression:} $g'(x) \le \frac{1}{2}(x-1)^2 \iff \frac{(x-1)^2}{2x^2} \le \frac{1}{2}(x-1)^2$
  \item \textbf{Justification:} Division par $\textbackslash{}frac\textbackslash{}{1\textbackslash{}}\textbackslash{}{2\textbackslash{}}(x-1)\textbackslash{}textasciicircum{}2$ (valide pour $x>1$, trivial pour $x=1$) \\
        \textbf{Expression:} $ \frac{1}{x^2} \le 1$ pour $x > 1$
  \item \textbf{Justification:} Multiplication par $x\textbackslash{}textasciicircum{}2$, toujours vrai pour $x \textbackslash{}ge 1$ \\
        \textbf{Expression:} $1 \le x^2$
\end{enumerate}
\textbf{Final Answer:} \boxed{$\text{Les dérivées et les encadrements sont prouvés.}$}

\subsection*{Question I.3}
\textbf{Givens:}
\begin{itemize}
  \item $f(1)=0$
  \item $g(1)=0$
  \item $0 \textbackslash{}le f'(x) \textbackslash{}le (x-1)\textbackslash{}textasciicircum{}2$
  \item $0 \textbackslash{}le g'(x) \textbackslash{}le \textbackslash{}frac\textbackslash{}{1\textbackslash{}}\textbackslash{}{2\textbackslash{}}(x-1)\textbackslash{}textasciicircum{}2$
\end{itemize}
\textbf{Requested:} $0 \textbackslash{}le f(x) \textbackslash{}le \textbackslash{}frac\textbackslash{}{1\textbackslash{}}\textbackslash{}{3\textbackslash{}}(x-1)\textbackslash{}textasciicircum{}3$ et $0 \textbackslash{}le g(x) \textbackslash{}le \textbackslash{}frac\textbackslash{}{1\textbackslash{}}\textbackslash{}{6\textbackslash{}}(x-1)\textbackslash{}textasciicircum{}3$

\textbf{Strategy Outline:}
\begin{enumerate}
  \item Integrate the inequalities for $f'(x)$ and $g'(x)$ from $1$ to $x$.
  \item Use the given initial conditions $f(1)=0$ and $g(1)=0$.
\end{enumerate}
\textbf{Detailed Steps:}
\begin{enumerate}
  \item \textbf{Justification:} Intégration de l'encadrement de $f'(x)$ sur $[1, x]$ \\
        \textbf{Expression:} $\int_1^x 0 \, dt \le \int_1^x f'(t) \, dt \le \int_1^x (t-1)^2 \, dt$
  \item \textbf{Justification:} Calcul des intégrales \\
        \textbf{Expression:} $0 \le [f(t)]_1^x \le \left[ \frac{(t-1)^3}{3} \right]_1^x$
  \item \textbf{Justification:} Application du théorème fondamental du calcul \\
        \textbf{Expression:} $0 \le f(x) - f(1) \le \frac{(x-1)^3}{3} - \frac{(1-1)^3}{3}$
  \item \textbf{Justification:} Utilisation de $f(1)=0$ et simplification \\
        \textbf{Expression:} $0 \le f(x) \le \frac{1}{3}(x-1)^3$
  \item \textbf{Justification:} Intégration de l'encadrement de $g'(x)$ sur $[1, x]$ \\
        \textbf{Expression:} $\int_1^x 0 \, dt \le \int_1^x g'(t) \, dt \le \int_1^x \frac{1}{2}(t-1)^2 \, dt$
  \item \textbf{Justification:} Calcul des intégrales \\
        \textbf{Expression:} $0 \le [g(t)]_1^x \le \frac{1}{2} \left[ \frac{(t-1)^3}{3} \right]_1^x$
  \item \textbf{Justification:} Application du théorème fondamental du calcul \\
        \textbf{Expression:} $0 \le g(x) - g(1) \le \frac{1}{2} \left( \frac{(x-1)^3}{3} - \frac{(1-1)^3}{3} \right)$
  \item \textbf{Justification:} Utilisation de $g(1)=0$ et simplification \\
        \textbf{Expression:} $0 \le g(x) \le \frac{1}{6}(x-1)^3$
\end{enumerate}
\textbf{Final Answer:} \boxed{$\text{Les encadrements sont déduits.}$}

\subsection*{Question II.1}
\textbf{Givens:}
\begin{itemize}
  \item $x \textbackslash{}in ]1, +\textbackslash{}infty[$
  \item $x\textbackslash{}_0 = x$
  \item $x\textbackslash{}_\textbackslash{}{n+1\textbackslash{}} = \textbackslash{}sqrt\textbackslash{}{x\textbackslash{}_n\textbackslash{}}$
  \item $t\textbackslash{}_n = 2\textbackslash{}textasciicircum{}n T(x\textbackslash{}_n)$
  \item $s\textbackslash{}_n = 2\textbackslash{}textasciicircum{}n S(x\textbackslash{}_n)$
\end{itemize}
\textbf{Requested:} a) Limite de $(x\textbackslash{}_n)$ ; b) $t\textbackslash{}_n \textbackslash{}le t\textbackslash{}_\textbackslash{}{n+1\textbackslash{}}$ et $s\textbackslash{}_\textbackslash{}{n+1\textbackslash{}} \textbackslash{}le s\textbackslash{}_n$ ; c) Monotonie de $(t\textbackslash{}_n)$ et $(s\textbackslash{}_n)$

\textbf{Strategy Outline:}
\begin{enumerate}
  \item For a), find fixed points of $x\textbackslash{}_\textbackslash{}{n+1\textbackslash{}} = \textbackslash{}sqrt\textbackslash{}{x\textbackslash{}_n\textbackslash{}}$ and analyze monotonicity.
  \item For b), use the inequalities from Question I.1 with $x\textbackslash{}_n$ and $x\textbackslash{}_\textbackslash{}{n+1\textbackslash{}}$.
  \item For c), directly state monotonicity from b).
\end{enumerate}
\textbf{Detailed Steps:}
\begin{enumerate}
  \item \textbf{Justification:} Recherche des points fixes de la relation de récurrence \\
        \textbf{Expression:} $L = \sqrt{L} \implies L^2 = L \implies L(L-1) = 0$
  \item \textbf{Justification:} Résolution de l'équation \\
        \textbf{Expression:} $L=0$ ou $L=1$
  \item \textbf{Justification:} Analyse de la relation $x\textbackslash{}_\textbackslash{}{n+1\textbackslash{}} = \textbackslash{}sqrt\textbackslash{}{x\textbackslash{}_n\textbackslash{}}$ pour $x\textbackslash{}_n > 1$ \\
        \textbf{Expression:} $x_n > 1 \implies \sqrt{x_n} < x_n$
  \item \textbf{Justification:} La suite $(x\textbackslash{}_n)$ est décroissante \\
        \textbf{Expression:} $x_{n+1} < x_n$
  \item \textbf{Justification:} La suite est minorée par 1 \\
        \textbf{Expression:} $x_n > 1$ pour tout $n$
  \item \textbf{Justification:} Une suite décroissante et minorée converge vers son point fixe \\
        \textbf{Expression:} $\lim_{n \to +\infty} x_n = 1$
  \item \textbf{Justification:} Inégalité de la Question I.1 appliquée à $x\textbackslash{}_n$ \\
        \textbf{Expression:} $T(x_n) \le 2T(\sqrt{x_n})$
  \item \textbf{Justification:} Substitution de $x\textbackslash{}_\textbackslash{}{n+1\textbackslash{}} = \textbackslash{}sqrt\textbackslash{}{x\textbackslash{}_n\textbackslash{}}$ \\
        \textbf{Expression:} $T(x_n) \le 2T(x_{n+1})$
  \item \textbf{Justification:} Multiplication par $2\textbackslash{}textasciicircum{}n$ \\
        \textbf{Expression:} $2^n T(x_n) \le 2^{n+1} T(x_{n+1})$
  \item \textbf{Justification:} Définition de $t\textbackslash{}_n$ et $t\textbackslash{}_\textbackslash{}{n+1\textbackslash{}}$ \\
        \textbf{Expression:} $t_n \le t_{n+1}$
  \item \textbf{Justification:} Inégalité de la Question I.1 appliquée à $x\textbackslash{}_n$ \\
        \textbf{Expression:} $2S(\sqrt{x_n}) \le S(x_n)$
  \item \textbf{Justification:} Substitution de $x\textbackslash{}_\textbackslash{}{n+1\textbackslash{}} = \textbackslash{}sqrt\textbackslash{}{x\textbackslash{}_n\textbackslash{}}$ \\
        \textbf{Expression:} $2S(x_{n+1}) \le S(x_n)$
  \item \textbf{Justification:} Multiplication par $2\textbackslash{}textasciicircum{}n$ \\
        \textbf{Expression:} $2^{n+1} S(x_{n+1}) \le 2^n S(x_n)$
  \item \textbf{Justification:} Définition de $s\textbackslash{}_n$ et $s\textbackslash{}_\textbackslash{}{n+1\textbackslash{}}$ \\
        \textbf{Expression:} $s_{n+1} \le s_n$
  \item \textbf{Justification:} D'après $t\textbackslash{}_n \textbackslash{}le t\textbackslash{}_\textbackslash{}{n+1\textbackslash{}}$ \\
        \textbf{Expression:} $\text{La suite } (t_n)_{n \in \mathbb{N}} \text{ est croissante.}$
  \item \textbf{Justification:} D'après $s\textbackslash{}_\textbackslash{}{n+1\textbackslash{}} \textbackslash{}le s\textbackslash{}_n$ \\
        \textbf{Expression:} $\text{La suite } (s_n)_{n \in \mathbb{N}} \text{ est décroissante.}$
\end{enumerate}
\textbf{Final Answer:} \boxed{a) $\lim_{n \to +\infty} x_n = 1$ ; b) $t_n \le t_{n+1}$ et $s_{n+1} \le s_n$ ; c) $(t_n)$ est croissante et $(s_n)$ est décroissante.}

\subsection*{Question II.2}
\textbf{Givens:}
\begin{itemize}
  \item $f(x) = \textbackslash{}ln(x) - T(x)$
  \item $g(x) = S(x) - \textbackslash{}ln(x)$
  \item $0 \textbackslash{}le f(x) \textbackslash{}le \textbackslash{}frac\textbackslash{}{1\textbackslash{}}\textbackslash{}{3\textbackslash{}}(x-1)\textbackslash{}textasciicircum{}3$
  \item $0 \textbackslash{}le g(x) \textbackslash{}le \textbackslash{}frac\textbackslash{}{1\textbackslash{}}\textbackslash{}{6\textbackslash{}}(x-1)\textbackslash{}textasciicircum{}3$
  \item $x\textbackslash{}_n = x\textbackslash{}textasciicircum{}\textbackslash{}{1/2\textbackslash{}textasciicircum{}n\textbackslash{}}$
  \item $t\textbackslash{}_n = 2\textbackslash{}textasciicircum{}n T(x\textbackslash{}_n)$
  \item $s\textbackslash{}_n = 2\textbackslash{}textasciicircum{}n S(x\textbackslash{}_n)$
\end{itemize}
\textbf{Requested:} $t\textbackslash{}_n \textbackslash{}le \textbackslash{}ln(x) \textbackslash{}le s\textbackslash{}_n$

\textbf{Strategy Outline:}
\begin{enumerate}
  \item Use $f(x\textbackslash{}_n) \textbackslash{}ge 0$ and $g(x\textbackslash{}_n) \textbackslash{}ge 0$ from Question I.3.
  \item Substitute the definitions of $f(x\textbackslash{}_n)$ and $g(x\textbackslash{}_n)$ to get inequalities involving $\textbackslash{}ln(x\textbackslash{}_n)$.
  \item Relate $\textbackslash{}ln(x\textbackslash{}_n)$ to $\textbackslash{}ln(x)$ using $x\textbackslash{}_n = x\textbackslash{}textasciicircum{}\textbackslash{}{1/2\textbackslash{}textasciicircum{}n\textbackslash{}}$.
  \item Multiply by $2\textbackslash{}textasciicircum{}n$ to obtain the desired encadrement.
\end{enumerate}
\textbf{Detailed Steps:}
\begin{enumerate}
  \item \textbf{Justification:} Utilisation de l'inégalité $f(x) \textbackslash{}ge 0$ de I.3 \\
        \textbf{Expression:} $f(x_n) \ge 0 \implies \ln(x_n) - T(x_n) \ge 0 \implies T(x_n) \le \ln(x_n)$
  \item \textbf{Justification:} Utilisation de l'inégalité $g(x) \textbackslash{}ge 0$ de I.3 \\
        \textbf{Expression:} $g(x_n) \ge 0 \implies S(x_n) - \ln(x_n) \ge 0 \implies \ln(x_n) \le S(x_n)$
  \item \textbf{Justification:} Combinaison des deux inégalités précédentes \\
        \textbf{Expression:} $T(x_n) \le \ln(x_n) \le S(x_n)$
  \item \textbf{Justification:} Propriété des logarithmes et définition de $x\textbackslash{}_n$ \\
        \textbf{Expression:} $\ln(x_n) = \ln(x^{1/2^n}) = \frac{1}{2^n} \ln(x)$
  \item \textbf{Justification:} Substitution de $\textbackslash{}ln(x\textbackslash{}_n)$ \\
        \textbf{Expression:} $T(x_n) \le \frac{1}{2^n} \ln(x) \le S(x_n)$
  \item \textbf{Justification:} Multiplication par $2\textbackslash{}textasciicircum{}n$ \\
        \textbf{Expression:} $2^n T(x_n) \le \ln(x) \le 2^n S(x_n)$
  \item \textbf{Justification:} Définition de $t\textbackslash{}_n$ et $s\textbackslash{}_n$ \\
        \textbf{Expression:} $t_n \le \ln(x) \le s_n$
\end{enumerate}
\textbf{Final Answer:} \boxed{$t_n \le \ln(x) \le s_n$}

\subsection*{Question II.3}
\textbf{Givens:}
\begin{itemize}
  \item $t\textbackslash{}_n = 2\textbackslash{}textasciicircum{}n T(x\textbackslash{}_n)$
  \item $s\textbackslash{}_n = 2\textbackslash{}textasciicircum{}n S(x\textbackslash{}_n)$
  \item $T(x) = \textbackslash{}frac\textbackslash{}{x\textbackslash{}textasciicircum{}2 - 1\textbackslash{}}\textbackslash{}{x\textbackslash{}textasciicircum{}2 + 1\textbackslash{}}$
  \item $S(x) = \textbackslash{}frac\textbackslash{}{x\textbackslash{}textasciicircum{}2 - 1\textbackslash{}}\textbackslash{}{2x\textbackslash{}}$
  \item $\textbackslash{}lim\textbackslash{}_\textbackslash{}{n \textbackslash{}to +\textbackslash{}infty\textbackslash{}} x\textbackslash{}_n = 1$
\end{itemize}
\textbf{Requested:} a) $s\textbackslash{}_n - t\textbackslash{}_n = \textbackslash{}frac\textbackslash{}{2\textbackslash{}textasciicircum{}n (x\textbackslash{}_n - 1)\textbackslash{}textasciicircum{}2\textbackslash{}}\textbackslash{}{2x\textbackslash{}_n (x\textbackslash{}_n + 1)\textbackslash{}}$ ; b) $\textbackslash{}lim\textbackslash{}_\textbackslash{}{n \textbackslash{}to +\textbackslash{}infty\textbackslash{}} (s\textbackslash{}_n - t\textbackslash{}_n) = 0$

\textbf{Strategy Outline:}
\begin{enumerate}
  \item For a), substitute the definitions of $s\textbackslash{}_n$, $t\textbackslash{}_n$, $S(x\textbackslash{}_n)$, and $T(x\textbackslash{}_n)$ and simplify the expression. (Note: The problem statement's target expression for $s\textbackslash{}_n - t\textbackslash{}_n$ appears to be incorrect. The derivation will lead to a different, mathematically correct expression.)
  \item For b), use the derived expression for $s\textbackslash{}_n - t\textbackslash{}_n$ and the limit of $x\textbackslash{}_n$ to evaluate the limit.
\end{enumerate}
\textbf{Detailed Steps:}
\begin{enumerate}
  \item \textbf{Justification:} Définition de $s\textbackslash{}_n$ et $t\textbackslash{}_n$ \\
        \textbf{Expression:} $s_n - t_n = 2^n S(x_n) - 2^n T(x_n) = 2^n (S(x_n) - T(x_n))$
  \item \textbf{Justification:} Substitution des définitions de $S(x\textbackslash{}_n)$ et $T(x\textbackslash{}_n)$ \\
        \textbf{Expression:} $S(x_n) - T(x_n) = \frac{x_n^2 - 1}{2x_n} - \frac{x_n^2 - 1}{x_n^2 + 1}$
  \item \textbf{Justification:} Factorisation par $(x\textbackslash{}_n\textbackslash{}textasciicircum{}2 - 1)$ \\
        \textbf{Expression:} $S(x_n) - T(x_n) = (x_n^2 - 1) \left( \frac{1}{2x_n} - \frac{1}{x_n^2 + 1} \right)$
  \item \textbf{Justification:} Mise au même dénominateur \\
        \textbf{Expression:} $S(x_n) - T(x_n) = (x_n^2 - 1) \left( \frac{x_n^2 + 1 - 2x_n}{2x_n(x_n^2 + 1)} \right)$
  \item \textbf{Justification:} Factorisation de $x\textbackslash{}_n\textbackslash{}textasciicircum{}2 - 1$ et $x\textbackslash{}_n\textbackslash{}textasciicircum{}2 - 2x\textbackslash{}_n + 1$ \\
        \textbf{Expression:} $S(x_n) - T(x_n) = (x_n - 1)(x_n + 1) \frac{(x_n - 1)^2}{2x_n(x_n^2 + 1)}$
  \item \textbf{Justification:} Substitution dans l'expression de $s\textbackslash{}_n - t\textbackslash{}_n$ (Note: L'expression donnée dans la question II.3.a est incorrecte. Ceci est la dérivation correcte.) \\
        \textbf{Expression:} $s_n - t_n = \frac{2^n (x_n - 1)^3 (x_n + 1)}{2x_n(x_n^2 + 1)}$
  \item \textbf{Justification:} Énoncé de la limite à calculer \\
        \textbf{Expression:} $\lim_{n \to +\infty} (s_n - t_n) = \lim_{n \to +\infty} \frac{2^n (x_n - 1)^3 (x_n + 1)}{2x_n(x_n^2 + 1)}$
  \item \textbf{Justification:} Résultat de la Question II.1.a \\
        \textbf{Expression:} $\lim_{n \to +\infty} x_n = 1$
  \item \textbf{Justification:} Approximation de Taylor de $e\textbackslash{}textasciicircum{}u - 1 \textbackslash{}approx u$ pour $u \textbackslash{}to 0$ \\
        \textbf{Expression:} $x_n - 1 = x^{1/2^n} - 1 \approx \frac{\ln x}{2^n}$ pour $n \to +\infty$
  \item \textbf{Justification:} Substitution de l'approximation \\
        \textbf{Expression:} $(x_n - 1)^3 \approx \left( \frac{\ln x}{2^n} \right)^3 = \frac{(\ln x)^3}{2^{3n}}$
  \item \textbf{Justification:} Simplification \\
        \textbf{Expression:} $2^n (x_n - 1)^3 \approx 2^n \frac{(\ln x)^3}{2^{3n}} = \frac{(\ln x)^3}{2^{2n}}$
  \item \textbf{Justification:} Substitution des limites des termes et simplification \\
        \textbf{Expression:} $\lim_{n \to +\infty} (s_n - t_n) = \lim_{n \to +\infty} \frac{(\ln x)^3 / 2^{2n} \cdot (1+1)}{2 \cdot 1 \cdot (1^2+1)} = \lim_{n \to +\infty} \frac{2(\ln x)^3}{4 \cdot 2^{2n}} = \lim_{n \to +\infty} \frac{(\ln x)^3}{2^{2n+1}}$
  \item \textbf{Justification:} Le dénominateur tend vers l'infini, donc la fraction tend vers 0 \\
        \textbf{Expression:} $\lim_{n \to +\infty} (s_n - t_n) = 0$
\end{enumerate}
\textbf{Final Answer:} \boxed{a) $s_n - t_n = \frac{2^n (x_n - 1)^3 (x_n + 1)}{2x_n(x_n^2 + 1)}$ ; b) $\lim_{n \to +\infty} (s_n - t_n) = 0$}

\subsection*{Question II.4}
\textbf{Givens:}
\begin{itemize}
  \item $(t\textbackslash{}_n)$ est croissante
  \item $(s\textbackslash{}_n)$ est décroissante
  \item $t\textbackslash{}_n \textbackslash{}le \textbackslash{}ln(x) \textbackslash{}le s\textbackslash{}_n$
  \item $\textbackslash{}lim\textbackslash{}_\textbackslash{}{n \textbackslash{}to +\textbackslash{}infty\textbackslash{}} (s\textbackslash{}_n - t\textbackslash{}_n) = 0$
\end{itemize}
\textbf{Requested:} Nature (convergence et limite) des suites $(t\textbackslash{}_n)$ et $(s\textbackslash{}_n)$

\textbf{Strategy Outline:}
\begin{enumerate}
  \item List the properties of the sequences $(t\textbackslash{}_n)$ and $(s\textbackslash{}_n)$.
  \item Conclude that they are adjacent sequences.
  \item State the convergence and common limit based on the definition of adjacent sequences.
\end{enumerate}
\textbf{Detailed Steps:}
\begin{enumerate}
  \item \textbf{Justification:} D'après la Question II.1.c \\
        \textbf{Expression:} $\text{La suite } (t_n)_{n \in \mathbb{N}} \text{ est croissante.}$
  \item \textbf{Justification:} D'après la Question II.1.c \\
        \textbf{Expression:} $\text{La suite } (s_n)_{n \in \mathbb{N}} \text{ est décroissante.}$
  \item \textbf{Justification:} D'après la Question II.2 ($t\textbackslash{}_n \textbackslash{}le \textbackslash{}ln(x) \textbackslash{}le s\textbackslash{}_n$) \\
        \textbf{Expression:} $t_n \le s_n \text{ pour tout } n \in \mathbb{N}$
  \item \textbf{Justification:} D'après la Question II.3.b \\
        \textbf{Expression:} $\lim_{n \to +\infty} (s_n - t_n) = 0$
  \item \textbf{Justification:} Définition des suites adjacentes (monotonie opposée, $t\textbackslash{}_n \textbackslash{}le s\textbackslash{}_n$, et différence tendant vers 0) \\
        \textbf{Expression:} $\text{Les suites } (t_n) \text{ et } (s_n) \text{ sont adjacentes.}$
  \item \textbf{Justification:} Propriété des suites adjacentes \\
        \textbf{Expression:} $\text{Les suites adjacentes convergent vers la même limite.}$
  \item \textbf{Justification:} L'encadrement $t\textbackslash{}_n \textbackslash{}le \textbackslash{}ln(x) \textbackslash{}le s\textbackslash{}_n$ et la convergence vers une limite commune impliquent que cette limite est $\textbackslash{}ln(x)$ \\
        \textbf{Expression:} $\lim_{n \to +\infty} t_n = \ln(x) \text{ et } \lim_{n \to +\infty} s_n = \ln(x)$
\end{enumerate}
\textbf{Final Answer:} \boxed{$\text{Les suites } (t_n) \text{ et } (s_n) \text{ sont adjacentes et convergent toutes deux vers } \ln(x).$}

\end{document}